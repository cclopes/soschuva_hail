% ---
% RESUMOS
% ---

% resumo em português
\setlength{\absparsep}{18pt} % ajusta o espaçamento dos parágrafos do resumo
\begin{resumo}
 Esta dissertação analisou tempestades que produziram granizo na Região Metropolitana de Campinas com o objetivo de identificar fatores determinantes para a ocorrência de queda de granizo. De forma inédita, uma rede de detecção de granizo instalada na região permitiu a identificação e determinação das intensidades das tempestades entre 2016 e 2017. O ciclo de vida, estrutura microfísica e cinemática de casos específicos foram estudadas usando três radares meteorológicos instalados no estado de São Paulo e uma rede de detecção de raios, usando ferramentas como rastreamento de sistemas convectivos, identificação de hidrometeoros e recuperação de vento tridimensional por Multi-Doppler. Comparando com escalas de intensidade de granizo aplicadas ao continente europeu, os casos analisados apresentaram intensidade baixa, com granizo de no máximo 22,4 mm de diâmetro. O caso de 2017-03-14 apresentou o tempo de vida mais longo (6,2 h), queda de granizo em duas localidades (com diâmetro máximo de 11,8 mm) e atividade elétrica mais intensa (taxa máxima de 107 (31) flashes/min IC (CG)), enquanto que o caso de 2017-11-15, com tempo de vida mais curto (2,2 h), apresentou baixa atividade elétrica (total de 46 (20) flashes IC (CG)) porém com queda de granizo mais intensa. Todas as queda de granizo dos casos específicos citados anteriormente estão associadas a uma extensa coluna de granizo identificada pelo radar polarimétrico e correntes ascendentes de até 30 m/s antes do evento; o granizo maior no caso de 2017-11-15 possivelmente tem contribuição da precipitação na forma líquida (associada à correntes descendentes mais intensas) que previne a diminuição de tamanho do granizo ao mesmo tempo que contribui para o seu crescimento mesmo abaixo da base da nuvem.

 \textbf{Palavras-chave}: Granizo. Microfísica de Nuvens. Multi-Doppler. São Paulo.
\end{resumo}

% resumo em inglês
\begin{resumo}[Abstract]
 \begin{otherlanguage*}{english}
   ...

   \vspace{\onelineskip}
 
   \noindent 
   \textbf{Keywords}: ...
 \end{otherlanguage*}
\end{resumo}