% ---
% RESUMOS
% ---

% resumo em português
\setlength{\absparsep}{18pt} % ajusta o espaçamento dos parágrafos do resumo
\begin{resumo}
 Esta dissertação analisou tempestades que produziram granizo na Região Metropolitana de Campinas com o objetivo de identificar fatores determinantes para a produção e precipitação de granizo. De forma inédita, uma rede de detecção de granizo instalada na região permitiu a identificação e determinação das intensidades das tempestades entre 2016 e 2017. O ciclo de vida, estrutura microfísica e cinemática de casos específicos foram estudados usando três radares meteorológicos instalados no estado de São Paulo e uma rede de detecção de raios, com ferramentas como rastreamento de sistemas convectivos, identificação de hidrometeoros e recuperação de vento tridimensional por Multi-Doppler. Comparando com escalas de intensidade de granizo aplicadas ao continente europeu, os casos analisados apresentaram intensidade baixa, com granizo de no máximo $22,4\:mm$ de diâmetro. O caso de 2017-03-14 apresentou o tempo de vida mais longo ($6,2\:h$), queda de granizo em duas localidades (com diâmetro máximo de $11,8\:mm$) e atividade elétrica mais intensa (taxa máxima de 107 (31) $flashes\:min^{-1}$ IC (CG)), enquanto que o caso de 2017-11-15, com tempo de vida mais curto ($2,2\:h$), apresentou baixa atividade elétrica (total de 46 (20) \textit{flashes} IC (CG)) porém com queda de granizo mais intensa. Todas as quedas de granizo dos casos específicos citados anteriormente estão associadas a uma extensa coluna de granizo identificada pelo radar polarimétrico e correntes ascendentes de até $30\:ms^{-1}$ antes do evento; o granizo maior no caso de 2017-11-15 possivelmente tem contribuição da precipitação na forma líquida (associada à correntes descendentes mais intensas) que previne a diminuição de tamanho do granizo ao mesmo tempo que contribui para o seu crescimento mesmo abaixo da base da nuvem.

 \textbf{Palavras-chave}: Granizo. Microfísica de Nuvens. Multi-Doppler. São Paulo.
\end{resumo}

% resumo em inglês
\begin{resumo}[Abstract]
 \begin{otherlanguage*}{english}
   This dissertation analyzed hail producing storms on the Metropolitan Region of Campinas to identify key factors for hailfall occurence. For the first time, a hail detection network installed in the region allowed the identification and determination of thunderstorm intensity in the 2016-2017 period. The life cycle, microphysical structure and kinematics of specific cases were studied  using three meteorological radars installed in São Paulo state and a lightning detection network, with tools such as tracking of convective systems, hydrometeor identification and Multi-Doppler 3D wind retrieval. The analyzed cases had low hailfall intensity when compared with scales applied in Europe, with $22.4\:mm$ maximum hail diameter. The 2017-03-14 case presented the longest lifetime ($6.2\:h$), hailfall in two locations ($11.8\:mm$ maximum hail diameter) and the most intense lightning activity (107 (31) $flashes\:min^{-1}$ IC (CG) maximum rate), while the 2017-11-15 case, with a shorter lifetime ($2.2\:h$), presented low electrical activity (46 (20) \textit{flashes} IC (CG) total) with the most intense hailfall. All hailfall cases of the specific cases mentioned earlier are associated with a extensive hail column identified by the polarimetric radar and up to $30\:ms^{-1}$ updrafts before the events; the bigger hail in the 2017-11-15 case posibly had the contribution of liquid precipitation (associated with larger downdrafts) which prevents hail size decrease as well as contributes to its growth below the cloud base.

   \textbf{Keywords}: Hail. Cloud Microphysics. Multi-Doppler. São Paulo.
 \end{otherlanguage*}
\end{resumo}