% ---
% Inserir errata
% ---
%\begin{errata}
%Elemento opcional da \citeonline[4.2.1.2]{NBR14724:2011}. Exemplo:
%
%\vspace{\onelineskip}
%
%FERRIGNO, C. R. A. \textbf{Tratamento de neoplasias ósseas apendiculares com
%reimplantação de enxerto ósseo autólogo autoclavado associado ao plasma
%rico em plaquetas}: estudo crítico na cirurgia de preservação de membro em
%cães. 2011. 128 f. Tese (Livre-Docência) - Faculdade de Medicina Veterinária e
%Zootecnia, Universidade de São Paulo, São Paulo, 2011.
%
%\begin{table}[htb]
%\center
%\footnotesize
%\begin{tabular}{|p{1.4cm}|p{1cm}|p{3cm}|p{3cm}|}
%  \hline
%   \textbf{Folha} & \textbf{Linha}  & \textbf{Onde se lê}  & \textbf{Leia-se}  \\
%    \hline
%    1 & 10 & auto-conclavo & autoconclavo\\
%   \hline
%\end{tabular}
%\end{table}
%
%\end{errata}
% ---

% ---
% Inserir folha de aprovação
% ---

% Isto é um exemplo de Folha de aprovação, elemento obrigatório da NBR
% 14724/2011 (seção 4.2.1.3). Você pode utilizar este modelo até a aprovação
% do trabalho. Após isso, substitua todo o conteúdo deste arquivo por uma
% imagem da página assinada pela banca com o comando abaixo:
%
% \begin{folhadeaprovacao}
% \includepdf{folhadeaprovacao_final.pdf}
% \end{folhadeaprovacao}
%
%\begin{folhadeaprovacao}
%
%  \begin{center}
%    {\ABNTEXchapterfont\large\imprimirautor}
%
%    \vspace*{\fill}\vspace*{\fill}
%    \begin{center}
%      \ABNTEXchapterfont\bfseries\Large\imprimirtitulo
%    \end{center}
%    \vspace*{\fill}
%    
%    \hspace{.45\textwidth}
%    \begin{minipage}{.5\textwidth}
%        \imprimirpreambulo
%    \end{minipage}%
%    \vspace*{\fill}
%   \end{center}
%        
%   Trabalho aprovado. \imprimirlocal, 24 de novembro de 2012:
%
%   \assinatura{\textbf{\imprimirorientador} \\ Orientador} 
%   \assinatura{\textbf{Professor} \\ Convidado 1}
%   \assinatura{\textbf{Professor} \\ Convidado 2}
%   %\assinatura{\textbf{Professor} \\ Convidado 3}
%   %\assinatura{\textbf{Professor} \\ Convidado 4}
%      
%   \begin{center}
%    \vspace*{0.5cm}
%    {\large\imprimirlocal}
%    \par
%    {\large\imprimirdata}
%    \vspace*{1cm}
%  \end{center}
%  
%\end{folhadeaprovacao}
% ---

% ---
% Dedicatória
% ---
%\begin{dedicatoria}
%	\vspace*{\fill}
%	\centering
%	\noindent
%	\textit{ Este trabalho é dedicado às crianças adultas que,\\
%		quando pequenas, sonharam em se tornar cientistas.} \vspace*{\fill}
%\end{dedicatoria}
% ---

% ---
% Agradecimentos
% ---
\begin{agradecimentos}
   À minha orientadora, Prof\textordfeminine\:Dr\textordfeminine\:Rachel Ifanger Albrecht, por todas as oportunidades dentro e fora da pesquisa, contribuindo imensamente para o meu crescimento pessoal e profissional.
   À Fundação de Amparo à Pesquisa do Estado de São Paulo (FAPESP), sob o processo n\textordmasculine\:2017/06075-3 (vinculado ao processo n\textordmasculine\:2015/14497-0) pelo apoio financeiro para a realização da pesquisa e apresentação em congresso internacional.
   À Secretaria e Seção de Informática do Departamento de Ciências Atmosféricas (DCA-IAG) pela ajuda técnica e burocrática.
   À toda equipe do Projeto SOS-CHUVA pela dedicação na instalação, manutenção, desenvolvimento e divulgação que permitiram que esse trabalho pudesse ser feito.
   Ao Prof\textordmasculine\:Dr. Lawrence D. Carey, Msc. Sarah Stough e demais colegas do grupo de trabalho do Departamento de Ciências Atmosféricas da Universidade do Alabama em Huntsville (UAH) que me auxiliaram no processamento da recuperação de vento por Multi-Doppler durante a visita à universidade.
   Aos amigos e colegas de departamento, especialmente Isabela Siqueira, Victória Peli e Alan Rosales, pela companhia e apoio, principalmente nas (muitas) horas difíceis.
   E, acima de tudo, à minha família, Janice, Fernando e Yukio, por serem meu grande alicerce durante esses dois anos.
\end{agradecimentos}
% ---

% ---
% Epígrafe
% ---
\begin{epigrafe}
	\vspace*{\fill}
	\begin{flushright}
		\begin{figure}[hb]
			\includegraphics[width=0.5\columnwidth,right]{figs/cat_meme.jpg}
		\end{figure}
		\textit{Fonte: \citeonline{Vernessa2017}.}
%		\textit{``Não vos amoldeis às estruturas deste mundo, \\
%			mas transformai-vos pela renovação da mente, \\
%			a fim de distinguir qual é a vontade de Deus: \\
%			o que é bom, o que Lhe é agradável, o que é perfeito.\\
%			(Bíblia Sagrada, Romanos 12, 2)}
	\end{flushright}
\end{epigrafe}
% ---