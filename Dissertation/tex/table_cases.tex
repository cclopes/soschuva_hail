\begin{table}[htb]
	\IBGEtab{%
		\caption{Descrição dos casos selecionados para análise.}%
		\label{tabela_casos}
	}{%
		\begin{tabularx}{\textwidth}{cYYY}
			\toprule
			Caso & Descrição & Regiões Afetadas & Tipo de Severidade \\
			\midrule
			2016-12-25 & Condições instáveis na região levou à formação de diversos sistemas convectivos & Campinas, Vale do Paraíba, São Carlos & Rajadas de vento, granizo \\
			\midrule 
			2017-01-31 & Linha de Instabilidade formada a partir de condições de calor e umidade favoráveis & Sorocaba, Itu, Araraquara & Granizo \\
			\midrule 
			2017-03-14 & Aquecimento da superfície e convergência de umidade associada a uma frente fria no oceano favoreceu a formação de sistemas convectivos no centro do estado & Campinas, Indaiatuba, Jacareí & Granizo \\
			\midrule 
			2017-11-15 & Condições localmente favoráveis levaram à formação de sistemas convectivos isolados no centro do estado de SP & Indaiatuba, Bebedouro & Granizo \\
			\midrule 
			2017-11-16 & Escoamento do Jato de Baixos Níveis no centro-sul do país possibilitou condições de calor e umidade favoráveis para a formação de sistemas convectivos em todo o estado & Lorena, Ribeirão Preto, Campinas, São Paulo, Itapeva & Rajadas de vento, granizo \\
			\bottomrule
		\end{tabularx}%
	}{%
		\fonte{Adaptado de \url{https://topicssoschuva.blogspot.com.br/2017/03/summary-of-case-studies.html}.}%
	}
\end{table}
