\chapter{Referencial Teórico}\label{teoria}

\section{Processos de Formação do Granizo}

\section{Granizo e Eletrificação de Tempestades}\label{granizo_eletrificacao}

\section{Tempestades de Granizo na América do Sul}

\section{Usando Radares Meteorológicos para Estudar a Cinemática das Tempestades}

\citeonline{morgan1986thunderstorm} afirmam que o campo de vento tridimensional extraído de dois ou mais radares Doppler oferece a descrição mais realista do ambiente relacionado ao granizo, já que efeitos de advecção são mostradas com alto grau de detalhamento (TRADUÇÃO MUITO SIMILAR)