% ----------------------------------------------------------
% Introdução (exemplo de capítulo sem numeração, mas presente no Sumário)
% ----------------------------------------------------------
\chapter{Introdução}
% ----------------------------------------------------------

[sobre granizo, severidade no sudeste - CHUVA, Nascimento, etc]

[sobre granizo em SP - Ceped]

[sobre o SOS-CHUVA]
"Como parte do Projeto SOS-CHUVA (FAPESP 2015/14497-0), pela primeira vez no
Brasil, temos medições in-situ da precipitação de granizo (dezembro de 2016 até o presente). Essas
medidas são compostas de distribuição de tamanho de granizo capturado por uma rede de detecçãp
de granizo (hailpad) instalada na Região Metropolitana de Campinas (RMC). Informações de
granizo combinadas com dados volumétricos de radar e descargas elétricas permitem um estudo
detalhado da microfísica, cinemática e eletrificação de granizo produzindo tempestades tropicais
durante o experimento de campo SOS-CHUVA"


\section{Objetivos}

O objetivo deste trabalho é explorar aspectos físicos de tempestades tropicais que geram granizo, de forma a caracterizar seus processos de formação e intensificação. A partir da cinemática convectiva e atividade elétrica, identificaremos fatores determinantes para que tempestades tropicais produzam granizo com tamanho suficiente para precipitar.