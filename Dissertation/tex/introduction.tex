% ----------------------------------------------------------
% Introdução (exemplo de capítulo sem numeração, mas presente no Sumário)
% ----------------------------------------------------------
\chapter{Introdução}
% ----------------------------------------------------------

Tempestades de granizo, ou seja, que produzem granizo de tamanho suficiente para precipitar, são consideradas severas pelo Serviço Meteorológico Americano (\textit{National Weather Service}, NWS) quando o granizo possui tamanho maior ou igual a $19\:mm$ \cite{Moller2001c}. Esse tipo de evento meteorológico causa prejuízos em diferentes escalas dependendo de sua intensidade, afetando principalmente o setor agrícola, o que motiva uma série de estudos sobre a frequência e distribuição espacial dessas tempestades ao redor do planeta \cite{Dessens1994, Knight2001, Eccel2012, Berthet2013, Baldi2014, Jin2017, Martins2017} com o objetivo de melhorar sua previsibilidade. Por mais que trabalhos como \citeonline{Herman2018} mostrem avanços na previsão de tempestades de granizo, a sua grande variabilidade em termos de estrutura microfísica, atividade elétrica e cinemática dificulta uma caracterização geral desse tipo de tempestade.

Os processos microfísicos de crescimento do granizo que permitem que ele precipite ainda como granizo são influenciados por outros aspectos inerentes à tempestade e o ambiente em que ela se desenvolve. Ambientes favoráveis para o desenvolvimento de tempestades de granizo são comumente encontrados em latitudes médias, principalmente quando são mais severas \cite{Court1982}. No entanto, é possível observar tempestades de granizo em regiões tropicais de maiores altitudes, que podem causar danos severos localmente \cite{Morgan1976}.

No sudeste do Brasil, tempestades intensas com grandes acumulados de precipitação em geral são mais comuns do que tempestades severas, afetando principalmente grandes centros urbanos vulneráveis como a Região Metropolitana de São Paulo (RMSP) \cite{Vemado2016}. Ainda assim, tempestades que apresentam queda de granizo tem um alto impacto social e econômico: a partir de dados da Defesa Civil do estado de São Paulo entre 1991 e 2012, 23 casos de queda de granizo foram registrados, afetando mais de 38 mil pessoas e causando danos materiais principalmente em habitações \cite{Ceped2012}. Caracterizar estes eventos e melhorar sua previsibilidade é fundamental para estimar seu efeito na região e desenvolver políticas públicas de prevenção à este e outros desastres.

Como continuação do Projeto CHUVA no Vale do Paraíba (São Paulo) \cite{Machado2014}, o Projeto SOS-CHUVA (FAPESP 2015/14497-0) foi desenvolvido na Região Metropolitana de Campinas (RMC) entre 2016 e 2018, contemplando duas estações chuvosas, com foco na observação e desenvolvimento de pesquisa em previsão imediata de tempestades intensas. Além da instalação de um radar de dupla polarização e estações meteorológicas na região, uma rede de detecção de granizo foi instalada pela primeira vez no sudeste do Brasil, fornecendo medições \textit{in-situ} de granizo. Ao combinar essas informações com dados volumétricos de radar e descargas elétricas, é possível estudar de forma inédita os processos físicos de tempestades que produzem granizo na região.


\section{Objetivos}

O objetivo deste trabalho é explorar aspectos físicos de tempestades tropicais que geram granizo, de forma a caracterizar seus processos de formação e intensificação. A partir da cinemática convectiva e atividade elétrica, identificaremos fatores determinantes para que tempestades tropicais produzam granizo com tamanho suficiente para precipitar.