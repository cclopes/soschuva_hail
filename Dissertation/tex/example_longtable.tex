%\setcaptionmargin{1cm}
\scriptsize
\begin{longtable}{cp{6cm}p{3.5cm}p{2.5cm}}
\caption{Descrição dos casos selecionados para análise}\\
\hline \\[-3ex]
\multicolumn{1}{c}{Caso} &
\multicolumn{1}{c}{Descrição} &
\multicolumn{1}{c}{Regiões Afetadas} &
\multicolumn{1}{c}{Tipo de Severidade}

\\[0.5ex] \hline \hline
\\[-2ex]

\endfirsthead

\multicolumn{4}{c}{\footnotesize{{\slshape{{\tablename} \thetable{}}} - Conclusão}}\\[0.5ex]

\hline \\[-2ex]

\multicolumn{1}{c}{Caso} &
\multicolumn{1}{c}{Descrição} &
\multicolumn{1}{c}{Regiões Afetadas} &
\multicolumn{1}{c}{Tipo de Severidade}

\\[0.5ex] \hline
\\[-2ex]

\endhead

\multicolumn{4}{l}{{\footnotesize{Continua\ldots}}}\\
\endfoot
\hline

\multicolumn{4}{l}{{\footnotesize{Fonte: Adaptado de https://topicssoschuva.blogspot.com.br/2017/03/summary-of-case-studies.html.}}}\\

\endlastfoot

2016-12-25 & Condições instáveis na região levou à formação de diversos sistemas convectivos & Campinas, Vale do Paraíba, São Carlos & Rajadas de vento, granizo \\
2017-01-31 & Linha de Instabilidade & Sorocaba, Itu, Araraquara & Granizo \\
2017-03-14 & Chuva forte e queda de granizo entre Campinas e Indaiatuba e em Jacareí & Campinas, Indaiatuba, Jacareí & Granizo \\
2017-11-15 & Condições termodinâmicas favoráveis levaram à formação de sistemas convectivos concentrados no centro do estado de SP &Indaiatuba, Bebedouro & Granizo \\
2017-11-16 & Condições termodinâmicas favoráveis e um cavado em médios níveis favoreceram a formação de sistemas convectivos & Lorena, Ribeirão Preto, Campinas, São Paulo, Itapeva & Rajadas de vento, granizo \\

\label{tabela_cases}
\end{longtable}
