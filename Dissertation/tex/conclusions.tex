\chapter{Conclusões}\label{conclusoes}

Esta dissertação descreveu tempestades tropicais que produziram queda de granizo na Região Metropolitana de Campinas de forma a identificar características físicas determinantes para a formação de granizos precipitáveis. A partir de uma rede de detecção de granizo instalada na região pelo Projeto SOS-CHUVA, cinco casos ocorridos entre 2016 e 2017 foram selecionados. Além da intensidade da queda de granizo inferida pelos registros dos \textit{hailpads}, três aspectos foram explorados: o ciclo de vida e atividade elétrica através do rastreamento dos sistemas convectivos observados pelo radar Banda-S de São Roque com o algoritmo ForTraCC-Radar e da rede de detecção de raios BrasilDAT, respectivamente; a microfísica das tempestades no momento da queda de granizo através de um algoritmo de identificação de hidrometeoros aplicado ao radar de dupla polarização Banda-S da FCTH e; a cinemática das tempestades antes e durante a queda de granizo através da recuperação de vento tridimensional por Multi-Doppler usando dois ou três radares.

Os casos selecionados mostraram quedas de granizo de baixa intensidade dentro de escalas de intensidade das tempestades de granizo aplicadas no continente europeu. O caso de 2017-11-15 foi o mais intenso, com granizo máximo de $22,4\:mm$, enquanto que o caso de 2017-03-14, com o ciclo de vida mais longo ($6,2\:h$) e queda de granizo em duas localidades distintas (Cosmópolis e Indaiatuba), foi o menos intenso, com granizo máximo de $11,8\:mm$. A baixa atividade elétrica do caso mais intenso, principalmente comparado com o caso menos intenso, motivou o estudo dos aspectos físicos destes dois casos. 

O ciclo de vida mais longo do caso de 2017-03-14 se deve a vários processos de fusão e separação com outros sistemas convectivos que se formaram na região durante a tarde e noite. Isso também contribuiu para a alta atividade elétrica, com taxas de até 107 (31) $flashes\:min^{-1}$ IC (CG) entre a queda de granizo em Cosmópolis e em Indaiatuba. A queda de granizo em Cosmópolis foi associada a um núcleo convectivo intenso embebido em um sistema multicelular, com granizo até $10\:km$ de altura, massa de gelo de até $15\:gm^{-3}$ e correntes ascendentes de até $30\:ms^{-1}$ antes do evento; a queda de granizo em Indaiatuba foi associada a um núcleo convectivo (não tão intenso quanto o anterior) dentro de um sistema com poucas células intensas, mas ainda assim com granizo (e chuva) até $10\:km$ de altura, massa de gelo de até $15\:gm^{-3}$ e correntes ascendentes de até $25\:ms^{-1}$ durante o evento. Já no caso de 2017-11-15, o sistema convectivo apresentou ciclo de vida mais curto ($2,2\:h$), menor extensão horizontal e total de 46 (20) \textit{flashes} IC (CG) (quantidade muito menor que no caso de 2017-03-14). A queda de granizo em Indaiatuba foi associada a um núcleo convectivo intenso quase isolado (após separação) com granizo até $11\:km$ de altura, massa de gelo de $30\:gm^{-3}$ logo abaixo da base da nuvem e correntes ascendentes de até $20\:ms^{-1}$ antes do evento.

Os casos de 2017-03-14 e 2017-11-15 foram de distintas escalas e ciclos de vida, portanto era de se esperar diferenças na intensidade da queda de granizo. Porém, as características microfísicas das tempestades durante o evento foram similares, com uma camada verticalmente extensa de granizo e massa de gelo entre a base e a região de fase mista da nuvem, o que não explica a diferença de tamanho dos granizos entre um caso e outro. A cinemática dessas tempestades possivelmente explica essa diferença: correntes ascendentes intensas (pelo menos $15\:ms^{-1}$) estão presentes dentro da região de fase mista da nuvem principalmente antes da queda de granizo em ambos os casos, o que promove a formação e crescimento de granizo, enquanto que durante a queda de granizo as correntes descendentes são mais intensas no caso de 2017-11-15. Correntes descendentes mais intensas indicam maior precipitação de hidrometeoros, que pode contribuir para a queda de granizos maiores evitando a diminuição de tamanho deles por evaporação abaixo da base da nuvem - a precipitação na forma líquida aumenta a umidade entre a superfície e a base da nuvem, ou seja, aumenta a pressão de vapor (em comparação com os arredores) e consequentemente diminui a taxa de evaporação do granizo. Além disso, o conceito de competição desleal - granizos maiores possuem maior velocidade terminal, o que permite que eles coletem gotículas de nuvem antes dos granizos menores e cresçam ainda mais rápido \cite{Knight2001} - também possui um papel importante, já que a precipitação fornece mais gotas em uma distância maior para o crescimento do granizo. Assim, mesmo com a baixa atividade elétrica, houve precipitação de granizos maiores no caso de 2017-11-15 possivelmente por contribuição da precipitação na forma líquida.

A diversidade de casos analisados nesta dissertação mostrou que, assim como em latitudes médias, quedas de granizo podem ocorrer em uma variedade de tempestades. Casos com ciclo de vida mais longo, constituído por sistemas convectivos multicelulares com alta atividade elétrica, assim como casos com ciclo de vida curto e baixa atividade elétrica, estão associados a queda de granizo. Considerando os dois casos analisados mais detalhadamente, é possível delinear algumas características em comum, como aumento da atividade elétrica antes ou depois do evento, a presença de granizo em uma camada extensa dentro da nuvem e a atuação da corrente ascendente dentro da região de fase mista contribuindo para a formação e crescimento do granizo.

\section{Sugestões para Trabalhos Futuros}\label{sugestoes}

O uso de uma rede de detecção de granizo em uma região de transição entre regimes tropical e subtropical para o estudo de tempestades que produzem granizo se mostrou muito importante para identificar e analisar características físicas deste tipo de sistema convectivo. A partir dessa base de dados (constituída por cerca de 10 casos, incluindo os 5 casos analisados nesta dissertação), esperamos que outros trabalhos explorem os demais casos não só em relação às suas características físicas mas também comparando 
parâmetros derivados dos \textit{hailpads} e de radares e/ou satélites meteorológicos, como por exemplo a energia cinética, de forma a validar estimativas de granizo por radar ou satélite na região. Para isso, é importante investir em métodos mais sofisticados (manuais ou semi-automáticos) de medição das distribuições de tamanho de granizo nos \textit{hailpads} de forma a evitar discrepâncias como as encontradas nesta dissertação.

